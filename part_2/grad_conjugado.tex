\begin{frame}{Gradiente Conjugado}
	\vspace{-22mm}
	\begin{itemize}
		\item Suaviza as mudanças de direção abruptas do método do gradiente.
		\vspace{5mm}
		\item Adição do coeficiente de inércia $\beta$ conserva uma fração da direção anterior.
	\end{itemize}
	\vspace{5mm}
	\begin{equation}
	d^{k+1} = -g(x^{k+1}) + \beta_k d^k
	\end{equation}

\end{frame}

\begin{frame}{Gradiente Conjugado - coeficientes}
	O avanço $\alpha$ tem a seguinte fórmula:
		\begin{align}
			x^{k+1} &= x^k + \alpha_k d^k \\
			\tilde{f}(\alpha) &= f(x^k + \alpha d^k)
		\end{align}
		A função $\tilde{f}(\alpha)$ é minimizada utilizando a razão áurea.\\
		Já o coeficiente $\beta$ é uma relação entre o gradiente atual e o anterior.\\
		Segundo \textit{Polak-Rebière}:
		\begin{equation}
			\beta_k = \frac{||g^{k+1}|| - [g^+1]^T g^k}{||g^k||}
		\end{equation}

\end{frame}