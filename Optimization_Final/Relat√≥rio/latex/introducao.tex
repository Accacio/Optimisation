O trabalho detalhado neste relatório se baseia em analisar métodos numéricos ``pobres'' para realizar busca de mínimos de funções vetoriais e compará-los aos métodos ``nobres". Os métodos implementados foram:

\begin{itemize}
	\item Simplex ou Nelder-Mead
	\item Algoritmo Genético
\end{itemize}

Existem diversas variações do método genético para minimização de funções, e por isso foi escolhido um deles e detalhado na sua respectiva seção. 

Os métodos implementados são chamados de ``pobres'' porque não necessitam de cálculos de derivadas ou hessianas, mas trabalham utilizando a força computacional dos computadores modernos. O objetivo deste trabalho é implementar tais métodos e compará-los aos métodos ``nobres", que realizam cálculos rebuscados para alcançar o mínimo das funções e mostrar quais são as vantagens e desvantagens de cada método.

Além disso, foram testados os métodos construídos de acordo com um problema real, para que fosse possível compará-los entre si e entre os métodos construídos em trabalhos passados.