Após desenvolver os métodos descritos nas seções \ref{sec:fibo}, \ref{sec:aurea} e \ref{sec:interpol} como funções do MATLAB, foi implementada uma interface gráfica, tornando a utilização desses métodos mais amigável ao usuário.\\

\par Essa interface foi desenvolvida no próprio MATLAB, em um ambiente chamado \textit{GUIDE} (Graphical User Interface Development Environment) que permite a criação da janela do programa com poucos cliques, conforme ilustrado na figura \ref{fig:guide}. A programação dos elementos da interface é feita em um arquivo *.m que está ligado a interface gráfica, armazenada em uma figura do matlab (*.fig).

\begin{figure}[H]
	\begin{center}
		\includegraphics[width=10cm]{../gui/guide.png}   
		\caption{Ambiente de desenvolvimento de interface gráfica \textit{GUIDE}.}
		\label{fig:guide}
	\end{center}
\end{figure}

O programa desenvolvido foi denominado \textbf{Busca de Mínimos Iterativa} e pode ser visto sua janela inicial na figura \ref{fig:gui}.

\begin{figure}[H]
	\begin{center}
		\includegraphics[width=10cm]{../gui/gui.png}   
		\caption{Janela do programa de Busca de Mínimos Iterativa.}
		\label{fig:gui}
	\end{center}
\end{figure}

\par O usuário começa colocando a expressão da função que ele deseja minimizar, lembrando de pontuar as multiplicações com um asterisco (*). Deve se definir então, \textbf{o método} a ser utilizado; o \textbf{máximo de iterações}, para que o programa não rode indefinidamente em casos em que não haja convergência; a \textbf{tolerância} desejada para o resultado procurado; e o \textbf{intervalo} onde o mínimo será buscado. \\

\par Feito isso, basta clicar no botão \textbf{Calcular} e os resultados serão exibidos ao lado, o valor de x que dá o mínimo da função e o valor minímo nesse ponto. Além disso, também é exibido o \textbf{gráfico da função} dentro do intervalo pedido e o mínimo é destacado com um asterisco vermelho ({\color{red} *}). \\
