Como podemos ver pelos resultados da busca de mínimos em cada seção, não existe um método melhor que os outros em todos os aspectos, mas para certas funções, os métodos podem ser qualificados.
\\

\par Para a função $ f_1(x) = 3x^2 + 20x - 8 $, podemos dizer que todos os métodos convergiram para o valor de mínimo, porém o método da interpolação foi o mais rápido e com menos iterações utilizadas. Entre Fibonacci e Áurea, o segundo foi um pouco mais rápido.
\\

\par Para a função $ f_2(x) = xsin(x)cos(x) $, os resultados foram semelhantes aos da $ f_1(x) $, com a interpolação sendo a mais rápida. Os três métodos possuem algoritmos de construção diferentes, e por isso, podemos ver que com uma função com mais de um mínimo, como a $ f_2(x) $, cada método pode encontrar um mínimo diferente.
\\

\par Para a função $ f_3(x) = 5x $, podemos ver que o método da interpolação falha em encontrar um mínimo, porque a função escolhida só possui mínimo na restrição escolhida, e assim, o algoritmo da interpolação não consegue construir uma parábola para localizar este valor. Os outros dois métodos conseguem encontrar um valor, sendo o método da Razão Áurea um pouco mais rápido.
\\

\par Para a função $ f_4(x) = -e^{-\mid x \mid} $, o método da interpolação fez todas as iterações e não conseguiu chegar perto do valor do mínimo, isso se deve porque o intervalo inicial dado ao método foi não-simétrico, e por isso, o algoritmo de interpolação falhou em encontrar um valor satisfatório para esse número de iterações. Já os métodos de Fibonacci e Áurea convergiram, sendo o primeiro ligeiramente mais rápido.
\\

\par Podemos concluir que nenhum método é ideal para todas as funções existentes, mas pudemos perceber que, para certas funções, existem métodos melhores que outros. Além disso, o método da interpolação, que parece ser mais rápido que os outros dois, nem sempre converge, fazendo assim com que os métodos da Razão Áurea e de Fibonacci sejam mais seguros.