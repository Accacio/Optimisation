\documentclass{beamer}
\usepackage[utf8]{inputenc}
\usepackage{epstopdf}
\usetheme[pageofpages=/,% String used between the current page and the
                         % total page count.
          bullet=circle,% Use circles instead of squares for bullets.
          titleline=true,% Show a line below the frame title.
          alternativetitlepage=true,% Use the fancy title page.
          titlepagelogo=logo_afonso,% Logo for the first page.
          watermark=Minerva_UFRJ_Black_transparency,% Watermark used in every page.
          watermarkheight=90px,% Height of the watermark.
          watermarkheightmult=4,% The watermark image is 4 times bigger
                                % than watermarkheight.
          ]{Torino}
\usepackage[brazil]{babel}
\usepackage{amsmath}
\usepackage{amsfonts}
\usepackage{amssymb}
\usepackage{xcolor}
\usepackage{caption}
\usepackage{pgf}
\usepackage{eulervm}

\author{\textbf{Alunos: \newline
		Cayo Valsamis \newline
		Gabriel Pelielo \newline
		Rafael Accácio \newline
		Rodrigo Moysés}}
\title{\textbf{Introdução à Otimização \vspace{0.25cm} \newline 
		       Trabalho 1}}
\institute{Universidade Federal do Rio de Janeiro}
\date{01 de Dezembro, 2015}


\begin{document}

\setbeamertemplate{section in toc}[square]

	\AtBeginSection[]{
	\begin{frame}
		\frametitle{Sumário}
		\tableofcontents[currentsection]
	\end{frame}
}
	
	
\begin{frame}[t,plain]
	\titlepage
\end{frame}
	
\begin{frame}{Sumário}
	\tableofcontents
\end{frame}

%O trabalho detalhado neste relatório se baseia em analisar métodos numéricos para realizar busca de mínimos de funções vetoriais. Foram implementados cinco métodos diferentes de localização de mínimos:

\begin{itemize}
	\item Método da Descida Máxima (ou gradiente)
	\item Método do Gradiente Conjugado
	\item Método de Newton
	\item Método de Newton Modificado
	\item Método de Quase Newton
\end{itemize}

Todos os métodos foram implementados na plataforma \textit{MATLAB}, através de um programa de interface gráfica usado para escolher o método desejado e inserir alguns valores necessários para executar a busca. O objetivo do trabalho foi comparar esses métodos de acordo com os quesitos de tempo de execução, número necessário de iterações e por fim qualificá-los de acordo com cada função inserida.
%\tableofcontents
\section{Fibonacci}
O método de Fibonacci é um método iterativo utilizado para a localização do mínimo de funções. Esse método utiliza um intervalo $ I_k $ = [a b], escolhemos então dois pontos simétricos em relação ao centro. Chamando $ x_1 $=a e $ x_4 $=b os pontos escolhidos serão $ x_2 $ e $ x_3 $. como pode ser visto na \ref{fig:ab}

\begin{figure}[h]
	\begin{center}
		\includegraphics[width=8cm]{../fibonacci/intervalo_inicial.png}   
		\caption{Segmento de reta $ \overline{ab} $}
		\label{fig:ab}
	\end{center}
\end{figure}

O mecanismo de redução do intervalo é bem simples, usa-se o critério dos 2 pontos. Para determinar o próximo intervalo $ I_k+1 $ calcula-se o valor da função f em $ x_2 $ e $ x_3 $, resultando em $ f_2 $ e $ f_3 $. Se $ f_2 $ < $ f_3 $ então [$ x_1 $ $ x_3 $] será o novo intervalo (\ref{fig:x1x3}) e se $ f_2 $ > $ f_3 $, então retemos [$ x_2 $ $ x_4 $] (\ref{fig:x2x4}). No caso de igualdade entre $ f_2 $ e $ f_3 $ a escolha é indiferente. 

\begin{figure}[h]
	\begin{center}
		\includegraphics[width=8cm]{../fibonacci/x1x3.png}   
		\caption{Intervalo antigo e novo}
		\label{fig:x1x3}
	\end{center}
\end{figure}

\begin{figure}[h]
	\begin{center}
		\includegraphics[width=8cm]{../fibonacci/x2x4.png}   
		\caption{Intervalo antigo e novo}
		\label{fig:x2x4}
	\end{center}
\end{figure}

Criamos então o novo intervalo $ I_k+1 $, caso ele seja:
\begin{itemize}
	\item [$ x_1 $ $ x_3 $] $\rightarrow$ $ x_1 $ permanece $ x_1 $, $ x_3 $ se torna o novo $ x_4 $, $ x_2 $ se torna o novo $ x_3 $ e um novo $ x_2 $ é colocado simetricamente ao novo $ x_3 $.
	\item [$ x_2 $ $ x_4 $] $\rightarrow$ $ x_4 $ permanece $ x_4 $, $ x_2 $ se torna o novo $ x_1 $, $ x_3 $ se torna o novo $ x_2 $ e um novo $ x_3 $ é colocado simetricamente ao novo $ x_2 $.
\end{itemize}

Só nos resta agora explicar como definir a escolha inicial de $ x_2 $ e $ x_3 $. Esses valores são escolhidos utilizando um fator 0,7 < $\alpha$ < 0,8, então algebricamente calculamos $ x_2  =  \alpha_1  x_1  + (1 -  \alpha_1 ) x_4 $ e $ x_3  = (1 -  \alpha_1 ) x_1  +  \alpha_1  x_4 $.
O valor de $\alpha$ é proveniente da fórmula utilizada para determinar o k-ésimo número da frequência de Fibonacci, portanto o método ganha esse nome.
Sendo k o número de reduções necessárias em determinado problema, podemos determinar o valor de $\alpha$ através de:
$\alpha = \dfrac{2}{1+\sqrt{5}} \times \dfrac{1-p^k}{1-p^{k+1}}$ onde $p=\dfrac{1-\sqrt{5}}{1+\sqrt{5}}$

Agora, com todas essas informações, podemos finalmente demonstrar o algoritmo que realmente será utilizado em sua forma computacional iterativa:
\textbf{Dados:}
\begin{itemize}
	\item $f:\mathbb{R}\rightarrow\mathbb{R}$, uma função suave e unimodal em
	\item $I_1=[ab]\subset\mathbb{R}$, intervalo enquadrante inicial
	\item $n\in\mathbb{Z}$ número desejado de reduções
	\item $k\in\mathbb{Z}$ índice de Fibonacci ($=n+1$)
\end{itemize}
\textbf{Objetivo:} Encontrar $I_n \subset I_1$ que enquadre um mínimo de $f$
\textbf{Operações:}
\begin{enumerate}
	\item $p=\dfrac{1-\sqrt{5}}{1+\sqrt{5}}$, $\alpha = \dfrac{2}{1+\sqrt{5}} \times \dfrac{1-p^k}{1-p^{k+1}}$
	\item $i=1$
	\item $x_1=a;\; x_4=b;\; L_{ini}=b-a;$
	\item $x_2=\alpha x_1+(1-\alpha)x_4;\; f_2=f(x_2)$
	\item $x_3=\alpha x_4 + (1 - \alpha)x_1;\; f_3 = f(x_3)$
	\item $ f_2 < f_3 $
	
	\begin{itemize}
		\item $a=x_1;\; b=x_3;\; L_{fin}=b-a;$
		\item $i=n \rightarrow I_n = [ab] \rightarrow FIM$
		\item $\alpha = \dfrac{(L_{ini}-L_{fin})}{L_{fin}};\; i=i+1$
		\item volta a 3.
	\end{itemize}
	

	\begin{itemize}
		\item $f_2 \geq f_3$
		\item $a=x_2;\; b=x_4;\; L_{fin}=b-a;$
		\item $i=n \rightarrow I_n=[ab] \rightarrow FIM$
		\item $\alpha = \dfrac{(L{ini}-L_{fin})}{L_{fin}};\; i=i+1$
		\item volta a 3.
	\end{itemize}
\end{enumerate}

Com o algoritmo de Fibonacci construído, foram testadas quatro funções para efeitos de comparação:

\begin{itemize}
	\item $ f_1(x) = 3x^2 + 20x - 8 $
	\item $ f_2(x) = xsin(x)cos(x) $
	\item $ f_3(x) = 5x $
	\item $ f_4(x) = -e^{-\mid x \mid} $
\end{itemize}

\newpage

Os resultados para a função $ f_1(x) $ foram:

\begin{figure}[h]
	\begin{center}
		\includegraphics[width=13cm]{../fibonacci/f1_gui.png}   
		\caption{Janela de inicialização de $ f_1(x) $}
		\label{fig:fibonacci-f1-gui}
	\end{center}
\end{figure}

\begin{figure}[h!]
	\begin{center}
		\includegraphics[width=6cm]{../fibonacci/f1_resultados.png}   
		\caption{Resultados detalhados de $ f_1(x) $}
		\label{fig:fibonacci-f1-resultados}
	\end{center}
\end{figure}

Os resultados para a função $ f_2(x) $ foram:

\begin{figure}[h]
	\begin{center}
		\includegraphics[width=13cm]{../fibonacci/f2_gui.png}   
		\caption{Janela de inicialização de $ f_2(x) $}
		\label{fig:fibonacci-f2-gui}
	\end{center}
\end{figure}

\begin{figure}[h!]
	\begin{center}
		\includegraphics[width=6cm]{../fibonacci/f2_resultados.png}   
		\caption{Resultados detalhados de $ f_2(x) $}
		\label{fig:fibonacci-f2-resultados}
	\end{center}
\end{figure}

Os resultados para a função $ f_3(x) $ foram:

\begin{figure}[h]
	\begin{center}
		\includegraphics[width=13cm]{../fibonacci/f3_gui.png}   
		\caption{Janela de inicialização de $ f_3(x) $}
		\label{fig:fibonacci-f3-gui}
	\end{center}
\end{figure}

\begin{figure}[h!]
	\begin{center}
		\includegraphics[width=6cm]{../fibonacci/f3_resultados.png}   
		\caption{Resultados detalhados de $ f_3(x) $}
		\label{fig:fibonacci-f3-resultados}
	\end{center}
\end{figure}

Os resultados para a função $ f_4(x) $ foram:

\begin{figure}[h]
	\begin{center}
		\includegraphics[width=13cm]{../aurea/f4_gui.png}   
		\caption{Janela de inicialização de $ f_4(x) $}
		\label{fig:fibonacci-f4-gui}
	\end{center}
\end{figure}

\begin{figure}[h!]
	\begin{center}
		\includegraphics[width=6cm]{../aurea/f4_resultados.png}   
		\caption{Resultados detalhados de $ f_4(x) $}
		\label{fig:fibonacci-f4-resultados}
	\end{center}
\end{figure}



\section{Razão Áurea}

\begin{frame}[t]{Áurea}
	O método de localização de mínimo da seção áurea é um aprimoramento do método de Fibonacci e faz uso da razão áurea,
	
	\begin{equation}
		R = \dfrac{\sqrt{5}-1}{2} \simeq 0.618
	\end{equation}
	
	\begin{figure}[h]
		\begin{center}
			\includegraphics[width=6cm]{./aurea_ex.png}   
			\caption{Algoritmo de seção áurea}
			\label{fig:aurea_ex}
		\end{center}
	\end{figure}
\end{frame}

\section{Interpolação Polinomial}
\begin{frame}[t]{Polinomial}
\begin{itemize}
	\item Método de minimização a partir de interpolação polinomial e minimização do polinômio encontrado.
	\vspace{1mm}
	\item Resolução do sistema: \\
	\begin{equation}
		A\vec{x} =\vec{b}
	\end{equation}
	\item $A$ é a matriz com os valores parciais da função, $\vec{x}$ são os coeficientes do polinômio e $\vec{b}$ os valores da função. Troca-se pior dos pontos iniciais pelo valor encontrado na interpolação. E calcula-se outro polinômio até convergir (até o intervalo ficar menor que a tolerência).
	
\end{itemize}
\end{frame}


\section{Programa}
\begin{frame}{Programa}
	\begin{block}{}
		\begin{itemize}
			\item Interface desenvolvida usando o ambiente \textit{GUIDE} do MATLAB.
			\vspace{1mm}
			\item Métodos de minimização vetorial feitos na forma de funções.
		\end{itemize}
		\vspace{-3mm}
		\begin{figure}[H]
			\begin{center}
				\includegraphics[width=5cm]{GUIDE}   
				\vspace{-0.15cm}
				\caption{Ambiente de desenvolvimento de interface gráfica}
				\label{fig:GUIDE}
			\end{center}
		\end{figure}
		
	\end{block}
\end{frame}
\begin{frame}{}
	\vspace{0mm}
	\hspace{4mm}
	\includegraphics[scale=0.55]{GUI}
\end{frame}

\section{Conclusão}
Como podemos ver pelos resultados da busca de mínimos em cada seção, não existe um método melhor que os outros em todos os aspectos, mas para certas funções, os métodos podem ser qualificados.
\\

\par Para a função $ f_1(x) = 3x^2 + 20x - 8 $, podemos dizer que todos os métodos convergiram para o valor de mínimo, porém o método da interpolação foi o mais rápido e com menos iterações utilizadas. Entre Fibonacci e Áurea, o segundo foi um pouco mais rápido.
\\

\par Para a função $ f_2(x) = xsin(x)cos(x) $, os resultados foram semelhantes aos da $ f_1(x) $, com a interpolação sendo a mais rápida. Os três métodos possuem algoritmos de construção diferentes, e por isso, podemos ver que com uma função com mais de um mínimo, como a $ f_2(x) $, cada método pode encontrar um mínimo diferente.
\\

\par Para a função $ f_3(x) = 5x $, podemos ver que o método da interpolação falha em encontrar um mínimo, porque a função escolhida só possui mínimo na restrição escolhida, e assim, o algoritmo da interpolação não consegue construir uma parábola para localizar este valor. Os outros dois métodos conseguem encontrar um valor, sendo o método da Razão Áurea um pouco mais rápido.
\\

\par Para a função $ f_4(x) = -e^{-\mid x \mid} $, o método da interpolação fez todas as iterações e não conseguiu chegar perto do valor do mínimo, isso se deve porque o intervalo inicial dado ao método foi não-simétrico, e por isso, o algoritmo de interpolação falhou em encontrar um valor satisfatório para esse número de iterações. Já os métodos de Fibonacci e Áurea convergiram, sendo o primeiro ligeiramente mais rápido.
\\

\par Podemos concluir que nenhum método é ideal para todas as funções existentes, mas pudemos perceber que, para certas funções, existem métodos melhores que outros. Além disso, o método da interpolação, que parece ser mais rápido que os outros dois, nem sempre converge, fazendo assim com que os métodos da Razão Áurea e de Fibonacci sejam mais seguros.

\end{document}

