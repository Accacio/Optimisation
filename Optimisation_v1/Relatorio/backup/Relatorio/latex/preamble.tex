%\usepackage{enumerate}
\usepackage{graphicx}
\usepackage{color}
\usepackage[cmex10]{amsmath}
\usepackage{array}
\usepackage{float}
\usepackage[utf8]{inputenc} 
\usepackage[portuguese]{babel}
\usepackage[font=normalsize,format=plain,labelfont=bf,up,textfont=up,figurename=Figura,tablename=Tabela]{caption}
\usepackage{subcaption}
\usepackage[top=1in, bottom=1in, left=1.25in, right=1.25in]{geometry}
\usepackage{indentfirst}
\usepackage{fancyhdr}

%% LaTeX Draw
%
%\usepackage[usenames,dvipsnames]{pstricks}
%\usepackage{epsfig}
%\usepackage{pst-grad} % For gradients
%\usepackage{pst-plot} % For axes
%\usepackage[space]{grffile} % For spaces in paths
%\usepackage{etoolbox} % For spaces in paths
%\makeatletter % For spaces in paths
%\patchcmd\Gread@eps{\@inputcheck#1 }{\@inputcheck"#1"\relax}{}{}
%\makeatother

% Font packages
\usepackage{amssymb}
\usepackage{amsfonts}
\usepackage{steinmetz}
% Nice extra font package, e.g. \mathds{1}
\usepackage{dsfont}


% Use multiple rows when writing tables
\usepackage{multirow}
\usepackage{booktabs}
\usepackage{bigstrut}
    \setlength\bigstrutjot{3pt}

% Uncomment next line to make footnots per page
\usepackage{perpage}

% Uncoment next group of lines to create the table of contents for the PDF
\usepackage{hyperref}
\definecolor{darkblue}{rgb}{0,0,0.5}
\hypersetup{
    pdftitle={Trabalho 1},
    pdfauthor={Gabriel Pelielo},
    bookmarksnumbered=true,     
    bookmarksopen=true,         
    bookmarksopenlevel=1,       
    colorlinks=true,
    linkcolor=darkblue,
    filecolor=darkblue,  
    urlcolor=darkblue,  
    citecolor=darkblue,              
    pdfstartview=Fit,           
    pdfpagemode=UseOutlines,    % this is the option you were lookin for
    pdfpagelayout=TwoPageRight
}

\renewcommand{\title}{Trabalho 1}
\newcommand{\subtitle}{Introdução à Otimização}
\pagestyle{fancy}
\fancyhead[L]{\title}
\fancyhead[R]{\subtitle}
\fancyhead[C]{\thepage}
\fancyfoot[C]{}

\allowdisplaybreaks