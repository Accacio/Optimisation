\begin{frame}[t]{Simplex}
\begin{itemize}
\item Método de minimização multidimensional 
sem restrições
\item John A. Nelder e Roger Mead, 1965 ``The Computer Journal''.
\item Utiliza um Simplex para minimizar uma função de $n$ variáveis. 
\end{itemize}

\note{Também chamado de método Nelder-Mead}
\end{frame}

\begin{frame}[t]{Simplex}
\begin{itemize}
\pause
\item O que é um Simplex?\pause
\end{itemize}
\begin{figure}[h]
	\begin{center}	
		\includegraphics[width=10cm]{simplex/simplex.pdf}
		\caption{Exemplos de Simplex para $\mathbb{R}^2$ e $\mathbb{R}^3$.}
		\label{fig:simplex}
	\end{center}
\end{figure}
\note<1>{Mas a pergunta que não quer calar? O que é um simplex?}
\note<2>{Um Simplex é o menor politopo possível para um espaço de n variáveis. Como exemplo, vemos a figura \ref{fig:simplex}. Para $\mathbb{R}^2$ o menor politopo é um Triângulo, e para $\mathbb{R}^3$ é um Tetraedro.}
\note<3>{Exemplos de simplex}
\end{frame}

\begin{frame}[t]{Simplex}
\begin{itemize}
\item Método Generalista\pause
\item Não Precisa de cálculos complexos\pause
\item Considerado método de ordem $0$
\end{itemize}
\note<1,2>{Este é um método bem generalista que serve para diversos problemas pela sua facilidade, e por não necessitar de Gradientes e Hessianas, como veremos a seguir, é considerado um método de ordem 0.}
\note<3>{Então para facilitar a visualização, representação e entendimento, será explicado o método para $\mathbb{R}^2$}
\end{frame}

\begin{frame}[t]{Divisão do Método}
\pause
\begin{enumerate}
\item Ordenação \pause
\item Busca do Centróide\pause 
\item Reflexão \pause
\item Expansão\pause
\item Contração\pause
\item Encolhimento
\end{enumerate}
\note<7>{Serão explicados nos próximos slides}
\end{frame}

\begin{frame}[t]{Ordenação}
\begin{equation}
x_l=min(f(x_i))
\end{equation}
\begin{equation}
x_h=max(f(x_i)), x_i\neq x_l
\end{equation}
\begin{center}
		\includegraphics[width=4cm]{simplex/simplex3}
		\includegraphics<2>[width=4cm]{simplex/simplex2}
		
		
		\only<1>{{\usebeamercolor[fg]{caption name}Figura:} Pontos antes da ordenação}
		\only<2>{{\usebeamercolor[fg]{caption name}Figura:} Pontos após ordenação}
\end{center}
\note{\begin{itemize}
\item $x_h$ high
\item $x_l$ low
\item outro $x_s$ arbitrariamente
\end{itemize}}
\end{frame}

\begin{frame}[t]{Busca do Centróide}
\begin{equation}
c=\dfrac{x_l+x_s}{2}
\end{equation}
\begin{center}
		\includegraphics<1>[width=4cm]{simplex/simplex2}
		\includegraphics<2>[width=4cm]{simplex/centroide}
\end{center}
\note{encontra-se o centróide entre todos os pontos excluindo o $x_h$}
\end{frame}

\begin{frame}[t]{Reflexão}
\begin{equation}
x_r=c+\alpha(c-x_h)
\end{equation}
\begin{itemize}
\item $\alpha$=1
\end{itemize}
	
\begin{center}
		\includegraphics<1>[width=4cm]{simplex/centroide}
		\includegraphics<2>[width=4cm]{simplex/reflexao.pdf}
				
		\only<2>{{\usebeamercolor[fg]{caption name}Figura:} Reflexão}
\end{center}	
\end{frame}

\begin{frame}[t]{Reflexão}
\begin{itemize}
\item Se $f(x_l)<f(x_r)<f(x_s)$, $x_h$:=$x_c$\pause
\item Caso contrário, realizar alguma das próximas transformações 
\end{itemize}
\note<1>{Coeficiente de reflexão}
\end{frame}

\begin{frame}{Expansão}
\begin{itemize}
\item Se $f(x_r)<f(x_l)$
\end{itemize}
\begin{equation}
x_e=c+\gamma(x_r-c)
\end{equation}
\begin{itemize}
\item $\gamma$=2
\end{itemize}
\begin{figure}[H]
	\begin{center}	
		\includegraphics[width=6cm]{simplex/expansao.pdf}
		\caption{Expansão.}
		\label{fig:expansao}
	\end{center}
\end{figure}
\note{Coeficiente de expansão}
\end{frame}

\begin{frame}{Expansão}
\begin{itemize}
\item Se $f(x_r)<f(x_e)$, $x_h$:=$x_r$
\item Se $f(x_e)<f(x_r)$, $x_h$:=$x_e$
\end{itemize}
\note{pega o menor entre eles}
\end{frame}

\begin{frame}[t]{Contração}
\begin{itemize}
\item Se $f(x_r)\geq f(x_l)$
\end{itemize}
\begin{subnumcases}{x_c=}
   c+\beta(x_r-c) & Se $f(x_s)\leq f(x_r)<f(x_h)$ \label{positive}
   \\
   c+\beta(x_h-c) & Se $f(x_r)>f(x_h)$ \label{negative}
\end{subnumcases}
\begin{itemize}
\item $\beta$=$\frac{1}{2}$
\end{itemize}
\begin{figure}[h]
	\begin{center}	
		\includegraphics[width=8cm]{simplex/contracao.pdf}
		\caption{Representação das Contrações.}
		\label{fig:contracao}
	\end{center}
\end{figure}
\note<1>{Coeficiente de contração}
\end{frame}

\begin{frame}[t]{Contração}
\[
\text{Contração para fora}
  \begin{cases} 
    \text{Substituir $x_h$ por $x_c$} & \text{Se } f(x_c)\leq f(x_r) \\
   \text{Realizar encolhimento}       & \text{Se } f(x_c)>f(x_r)
  \end{cases}\]
  \[
  \text{Contração para dentro}
  \begin{cases} 
   \text{Substituir $x_h$ por $x_c$} & \text{Se } f(x_c)<f(x_h) \\
   \text{Realizar encolhimento}       & \text{Se } f(x_c)\geq f(x_h)
  \end{cases}
\]
\end{frame}


\begin{frame}[t]{Encolhimento}
\begin{subequations}
\begin{align}
x_s:=x_s+\delta (x_s-x_l)\\
x_h:=x_h+\delta (x_h-x_l)
\end{align}
\end{subequations}
\begin{itemize}
\item $\delta$=$\frac{1}{2}$
\end{itemize}
\begin{figure}[h]
	\begin{center}	
		\includegraphics[width=4cm]{simplex/encolher.pdf}
		\caption{Encolhimento.}
		\label{fig:encolher}
	\end{center}
\end{figure}
\note{O encolhimento é muito raro de acontecer e é usado para quando todas as outras transformações anteriores resultam em pontos cujo valor correspondente na função é pior que os anteriores. O que só acontece em casos específicos. Exemplo de caso pode ser visto um extrato da publicação original de 1965
}
\end{frame}

\begin{frame}[t]{Encolhimento}
\begin{quote}
``A failed contraction is much rarer, but can occur when a valley is curved and one point of the simplex is much farther from the valley bottom than the others; contraction may then cause the reflected point to move away from the valley bottom instead of towards it. Further contractions are then useless. The action proposed contracts the simplex towards the lowest point, and will eventually bring all points into the valley.''
\end{quote}
\note{blablablabla....}
\end{frame}

\begin{frame}[t]{Critérios de Parada}
\begin{enumerate}\pause
\item Iterações\pause
\item Raio da circunferência circunscrita
\item<4-5> Desvio Padrão dos $f(x)$
\end{enumerate}
\begin{center}
		\includegraphics<3>[width=5cm]{simplex/parada.pdf}
		\includegraphics<4-5>[width=8cm]{simplex/desviopadrao.pdf}
				
		\only<3>{{\usebeamercolor[fg]{caption name}Figura:} Circunferência circunscrita ao simplex.}
		\only<4>{{\usebeamercolor[fg]{caption name}Figura:} Desvio padrão}
		\only<5>{{\usebeamercolor[fg]{caption name}} Onde $\sigma$=$\sqrt{\frac{\sum_i^n(y_i-\overline{y})^2}{n}}$}
\end{center}
\note<1>{Foram utilizados 3 critérios de parada nesse método}
\note<4>{Cálculo do desvio utilizando três dados por vez. Cada vértice do triângulo a partir da fórmula a seguir}
\note<5>{Quando o desvio chega a zero significa que a superfície se comporta como um plano que não possui mínimo.}
\end{frame}
