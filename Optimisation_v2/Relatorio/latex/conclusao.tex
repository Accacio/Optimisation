Os valores encontrados na seção \ref{sec:resultados} para a minimização do exercício proposto, quando comparados com os valores reais ($ \omega_n = 5.0 $  e  $ \zeta = 0.28 $) encontram-se bem próximos, com discrepâncias menores que $10\%$. 

Esses resultados evidenciam o poder dos métodos pobres, que apesar de geralmente exigirem um alto poder computacional, não mostram dificuldade em resolver problemas que os métodos nobres não conseguem, por necessitar de um cálculo simbólico complexo de derivadas e hessianas, o que resulta em um tempo de processamento muito grande.



