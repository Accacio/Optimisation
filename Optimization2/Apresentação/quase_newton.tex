\begin{frame}{Método de Quase Newton}
	Todos os métodos de Newton baseiam-se em facilitar o cálculo da Hessiana. O método de Quase Newton diferencia-se por apoiar-se na chamada "Condição de Quase Newton":
	\\
	\begin{equation}
		H^{k+1} \gamma^{k} = \delta^{k} = \left\{
		\begin{tabular}{c}
		\gamma^{k} = g^{k+1} - g^{k}\\
		\delta^{k} = x^{k+1} - x^{k} 
		\end{tabular}
	\end{equation}
\end{frame}

\begin{frame}	
Duas possibilidades para se gerar matrizes H satisfazendo essa restrição são o método de Davidon-Fletcher-Powell (DFP):
\begin{equation}
	H^{k+1} = H - \frac{H \gamma \gamma^T H}{\gamma^T H \gamma} + \frac{\delta \delta^T}{\delta^T \gamma}
\end{equation}	

ou o método de Broyden-Fletcher-Goldfarb-Shanno (BFGS)

\begin{equation}
H^{k+1} = H - \frac{\delta \gamma^T H + H \gamma \delta^T}{\delta^T \gamma} + \Bigg( 1 + \frac{\gamma^T H \gamma}{\delta^T \gamma}\Bigg) \frac{\delta \delta^T}{\delta^T \gamma}
\end{equation}	


\end{frame}