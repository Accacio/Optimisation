\documentclass{beamer}
\usepackage[utf8]{inputenc}
\usepackage{epstopdf}
\usetheme[pageofpages=/,% String used between the current page and the
                         % total page count.
          bullet=circle,% Use circles instead of squares for bullets.
          titleline=true,% Show a line below the frame title.
          alternativetitlepage=true,% Use the fancy title page.
          titlepagelogo=logo_afonso,% Logo for the first page.
          watermark=Minerva_UFRJ_Black_transparency,% Watermark used in every page.
          watermarkheight=90px,% Height of the watermark.
          watermarkheightmult=4,% The watermark image is 4 times bigger
                                % than watermarkheight.
          ]{Torino}
\usepackage[brazil]{babel}

\author{\textbf{Alunos: \newline
		Cayo Valsamis \newline
		Gabriel Pelielo \newline
		Rafael Accácio \newline
		Rodrigo Moysés}}
\title{\textbf{Introdução à Otimização \vspace{0.25cm} \newline 
		       Trabalho 2}}
\institute{Universidade Federal do Rio de Janeiro}
\date{19 de Janeiro, 2016}

\begin{document}
	
	\setbeamertemplate{section in toc}[square]
	
	\AtBeginSection[]{
		\begin{frame}
			\frametitle{Sumário}
			\tableofcontents[currentsection]
		\end{frame}
	}
	
	
	\begin{frame}[t,plain]
		\titlepage
	\end{frame}
	
	\begin{frame}{Sumário}
		\tableofcontents
	\end{frame}
	
	%O trabalho detalhado neste relatório se baseia em analisar métodos numéricos para realizar busca de mínimos de funções vetoriais. Foram implementados cinco métodos diferentes de localização de mínimos:

\begin{itemize}
	\item Método da Descida Máxima (ou gradiente)
	\item Método do Gradiente Conjugado
	\item Método de Newton
	\item Método de Newton Modificado
	\item Método de Quase Newton
\end{itemize}

Todos os métodos foram implementados na plataforma \textit{MATLAB}, através de um programa de interface gráfica usado para escolher o método desejado e inserir alguns valores necessários para executar a busca. O objetivo do trabalho foi comparar esses métodos de acordo com os quesitos de tempo de execução, número necessário de iterações e por fim qualificá-los de acordo com cada função inserida.
	%\tableofcontents
	
	
	\section{Descida Máxima}
	O método de Descida máxima, ou método do gradiente, se baseia em um método muito simples para encontrar o mínimo de uma função.
 A partir de um ponto inicial calcula-se a direção de maior de crescimento, de descida máxima ou sentido inverso do gradiente, que dão nome ao método, 
e faz-se um avanço nessa mesma direção.\\
 
Como sabemos, o gradiente de uma função representa sua direção de maior crescimento. Sabendo disso utiliza-se a direção oposta, 
mas a questão que fica é qual o avanço é necessário. Para resolver isso, a direção do gradiente foi normalizada, $d^k$, e criou-se uma variável, $\alpha$, que indica o avanço a ser feito. 

Com o gradiente calculado naquele ponto, usa-se um método de minimização unidimensional qualquer para minimizar a função $x^k - \alpha d^k$. 
Foi escolhido o método da seção áurea, devido a sua grande flexibilidade e por encontrar o mínimo com poucas iterações e baixo esforço computacional. 
Após a minimização utilizamos o valor de $\alpha$ que a minimiza como avanço.
\\
Dessa forma faz-se a iteração para encontrar o próximo ponto, que a princípio 
está mais perto do mínimo. E assim faz-se até convergir ao ponto de mínimo da função (considerando que a mesma o possui).

Como métodos de parada foram escolhidos o tamanho da norma do vetor diferença de gradiente em cada iteração, a norma entre a diferença entre os pontos em cada iteração e finalmente o número de iterações.

Para verificar a convergência do método vemos como são as direções de avanço do algoritmo e também sua norma. Sabe-se a partir de \cite{Notastioafel} que cada iteração tem direção perpendicular a anterior e que a convergência dá-se quando cada avanço é menor que o anterior, assim como vemos na figura \ref{fig:stegradeszigzag}


\begin{figure}[H]
	\begin{center}
		\includegraphics[width=8cm]{../tikz/stegrades}  
		\caption{Representação do avanço convergente em zig-zag com ângulos retos.}
		\label{fig:stegradeszigzag}
	\end{center}
\end{figure}


	\begin{quote}
		\centering
		Lei de Iteração:
	\end{quote}

\begin{equation}
	x^{k+1} = x^k - \alpha d^k
	\end{equation}
	
Após construir o algoritmo, foram feitos testes usando algumas funções:

\begin{itemize}
	\item $ f_1(x,y) = x^2 + y^2$
	\item $ f_2(x,y) = -e^{-x^2 -y^2}$
	\item $ f_3(x,y) = cos(\frac{xy}{5})+sin(\frac{xy}{5}) $
	\item $ f_4(x,y) = |x+y| $
\end{itemize}


\newpage

\begin{figure}[H]
	\begin{center}	
		\includegraphics[width=12cm]{../stegrades/f1_gui.PNG}
		\caption{Janela de inicialização de $ f_1(x,y) $}
		\label{fig:f1_gui}
	\end{center}
\end{figure}



\begin{figure}[H]
	\begin{center}	
		\includegraphics[width=12cm]{../stegrades/f2_gui.PNG}
		\caption{Janela de inicialização de $ f_2(x,y) $}
		\label{fig:f2_gui}
	\end{center}
\end{figure}



\begin{figure}[H]
	\begin{center}	
		\includegraphics[width=12cm]{../stegrades/f3_gui.PNG}
		\caption{Janela de inicialização de $ f_3(x,y) $}
		\label{fig:f3_gui}
	\end{center}
\end{figure}



\begin{figure}[H]
	\begin{center}	
		\includegraphics[width=12cm]{../stegrades/f4_gui.PNG}
		\caption{Janela de inicialização de $ f_4(x,y) $}
		\label{fig:f4_gui}
	\end{center}
\end{figure}






\newpage




	
	
	\section{Gradiente Conjugado}
	\begin{frame}{Gradiente Conjugado}
	\vspace{-22mm}
	\begin{itemize}
		\item Suaviza as mudanças de direção abruptas do método do gradiente.
		\vspace{5mm}
		\item Adição do coeficiente de inércia $\beta$ conserva uma fração da direção anterior.
	\end{itemize}
	\vspace{5mm}
	\begin{equation}
	d^{k+1} = -g(x^{k+1}) + \beta_k d^k
	\end{equation}

\end{frame}

\begin{frame}{Gradiente Conjugado - coeficientes}
	O avanço $\alpha$ tem a seguinte fórmula:
		\begin{align}
			x^{k+1} &= x^k + \alpha_k d^k \\
			\tilde{f}(\alpha) &= f(x^k + \alpha d^k)
		\end{align}
		A função $\tilde{f}(\alpha)$ é minimizada utilizando a razão áurea.\\
		Já o coeficiente $\beta$ é uma relação entre o gradiente atual e o anterior.\\
		Segundo \textit{Polak-Rebière}:
		\begin{equation}
			\beta_k = \frac{||g^{k+1}|| - [g^+1]^T g^k}{||g^k||}
		\end{equation}

\end{frame}
	
	\section{Newton}
	\begin{frame}{Método de Newton}
	Os métodos de Newton se baseiam em encontrar uma aproximação quadrática $ q(x) $ a partir do teorema de Taylor para a função objetivo $ f(x) $, e assim, encontrar o seu mínimo.
	
	\begin{quote}
		\centering
		Lei de Iteração:
	\end{quote}

	\begin{equation}
	x^{k+1} = x^k - [G^k]^{-1}g^k
	\end{equation}
	
	
\end{frame}	
	
	\section{Newton Modificado}
	\begin{frame}{Newton Modificado}
	
	Para melhorar o método de Newton, foram feitas algumas alterações, como:
	
	\begin{itemize}
		\item Diminuição do avanço na direção $ d^k $, da forma:
	\end{itemize}
	
	\begin{equation}
	x^{k+1} = x^k + \alpha_k d^k
	\end{equation}
	
	onde $ \alpha_k $ minimiza $ \tilde{f}(\alpha) = f(x^k + \alpha d^k) $
	
\end{frame}


\begin{frame}{Newton Modificado}
	
	\begin{itemize}
		\item Correção do sinal da Hessiana por um truque matricial:
	\end{itemize}	
	
	\begin{equation}
	F^k = G^k + \gamma I_n
	\end{equation}
	
	Em que $ F^k $ tem autovalores positivos para poder gerar uma direção $ d^k $ de descida da forma
	
	\begin{equation}
	d^k = -[F^k]^{-1}g^k = -[\nabla^2 f(x^k) + \gamma I_n]^{-1}\nabla f(x^k)
	\end{equation}
	
\end{frame}
	
	\section{Quase Newton}
	\begin{frame}{Método de Quase Newton}
	Todos os métodos de Newton baseiam-se em facilitar o cálculo da Hessiana. O método de Quase Newton diferencia-se por apoiar-se na chamada "Condição de Quase Newton":
	\\
	\begin{equation}
		H^{k+1} \gamma^{k} = \delta^{k} = \left\{
		\begin{tabular}{c}
		\gamma^{k} = g^{k+1} - g^{k}\\
		\delta^{k} = x^{k+1} - x^{k} 
		\end{tabular}
	\end{equation}
\end{frame}

\begin{frame}	
Duas possibilidades para se gerar matrizes H satisfazendo essa restrição são o método de Davidon-Fletcher-Powell (DFP):
\begin{equation}
	H^{k+1} = H - \frac{H \gamma \gamma^T H}{\gamma^T H \gamma} + \frac{\delta \delta^T}{\delta^T \gamma}
\end{equation}	

ou o método de Broyden-Fletcher-Goldfarb-Shanno (BFGS)

\begin{equation}
H^{k+1} = H - \frac{\delta \gamma^T H + H \gamma \delta^T}{\delta^T \gamma} + \Bigg( 1 + \frac{\gamma^T H \gamma}{\delta^T \gamma}\Bigg) \frac{\delta \delta^T}{\delta^T \gamma}
\end{equation}	


\end{frame}
	
	\section{Programa}
	\begin{frame}{Programa}
	\begin{block}{}
		\begin{itemize}
			\item Interface desenvolvida usando o ambiente \textit{GUIDE} do MATLAB.
			\vspace{1mm}
			\item Métodos de minimização vetorial feitos na forma de funções.
		\end{itemize}
		\vspace{-3mm}
		\begin{figure}[H]
			\begin{center}
				\includegraphics[width=5cm]{GUIDE}   
				\vspace{-0.15cm}
				\caption{Ambiente de desenvolvimento de interface gráfica}
				\label{fig:GUIDE}
			\end{center}
		\end{figure}
		
	\end{block}
\end{frame}
\begin{frame}{}
	\vspace{0mm}
	\hspace{4mm}
	\includegraphics[scale=0.55]{GUI}
\end{frame}
	
	\section{Conclusão}
	Como podemos ver pelos resultados da busca de mínimos em cada seção, não existe um método melhor que os outros em todos os aspectos, mas para certas funções, os métodos podem ser qualificados.
\\

\par Para a função $ f_1(x) = 3x^2 + 20x - 8 $, podemos dizer que todos os métodos convergiram para o valor de mínimo, porém o método da interpolação foi o mais rápido e com menos iterações utilizadas. Entre Fibonacci e Áurea, o segundo foi um pouco mais rápido.
\\

\par Para a função $ f_2(x) = xsin(x)cos(x) $, os resultados foram semelhantes aos da $ f_1(x) $, com a interpolação sendo a mais rápida. Os três métodos possuem algoritmos de construção diferentes, e por isso, podemos ver que com uma função com mais de um mínimo, como a $ f_2(x) $, cada método pode encontrar um mínimo diferente.
\\

\par Para a função $ f_3(x) = 5x $, podemos ver que o método da interpolação falha em encontrar um mínimo, porque a função escolhida só possui mínimo na restrição escolhida, e assim, o algoritmo da interpolação não consegue construir uma parábola para localizar este valor. Os outros dois métodos conseguem encontrar um valor, sendo o método da Razão Áurea um pouco mais rápido.
\\

\par Para a função $ f_4(x) = -e^{-\mid x \mid} $, o método da interpolação fez todas as iterações e não conseguiu chegar perto do valor do mínimo, isso se deve porque o intervalo inicial dado ao método foi não-simétrico, e por isso, o algoritmo de interpolação falhou em encontrar um valor satisfatório para esse número de iterações. Já os métodos de Fibonacci e Áurea convergiram, sendo o primeiro ligeiramente mais rápido.
\\

\par Podemos concluir que nenhum método é ideal para todas as funções existentes, mas pudemos perceber que, para certas funções, existem métodos melhores que outros. Além disso, o método da interpolação, que parece ser mais rápido que os outros dois, nem sempre converge, fazendo assim com que os métodos da Razão Áurea e de Fibonacci sejam mais seguros.
	
\end{document}